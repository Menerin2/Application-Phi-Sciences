\documentclass{article}
\usepackage[T1]{fontenc}
\usepackage[utf8]{inputenc}
\usepackage[french]{babel}
\usepackage{enumitem}
\usepackage{amsmath}
\usepackage{amssymb}
\usepackage{graphicx}
\usepackage{hyperref}
\hypersetup{
    colorlinks,
    citecolor=black,
    filecolor=black,
    linkcolor=black,
    urlcolor=black
}
\setcounter{tocdepth}{2}

\title{Cahier des charges :\\Application de gestion de stock}
\author{LOI Léo}

\begin{document}
\maketitle
\tableofcontents
\newpage

\section{Présentation du projet}
  Le projet est présenté sous la forme d’une application pour mobile permettant de gérer les stocks de l’association Phi-Sciences.

  Le stock est divisé en 3 grandes parties : 
  \begin{enumerate}
    \item Les consommables qui contiennent : 
    \begin{itemize}
      \item La nourriture et les boissons qui sont identifiées par leur nom, leur numéro de lot et leur date d'achat. Il faut, pour ces produits, garder une traçabilité de la date d'achat, de péremptionn et potentiellement d'ouverture mais aussi garder une trace de l'évolution de la quantité de chaque produit dans le temps
      \item Les produits d’entretien qui sont identifiées par leur nom et leur date d'achat. Il faut pour ces produits uniquement garder une trace de l'évolution de leur quantité dans le temps
      \item Les consommables autres (essuie-tout, assiettes en carton, ...) qui sont aussi identifiés par leur nom et leur date d'achat. Il faut pour ces produits aussi uniquement garder une trace de l'évolution de leur quantité dans le temps
    \end{itemize}
    \item Le matériel qui est identitfié par leur nom. Il faut pour le matériel garder une traçabilité de l'état du matériel et des différentes utilisations ou prêt du matériel
    \item Les goodies qui sont identifiées par leur nom. Il faut garder une traçabilité de la quantité de chaque goodies au fil du temps.
  \end{enumerate}
  En plus de ce qui ls identifies, chaque produit/matériel sera identifié par un identifiant alpha-numérique unique

  Tout membre actif de l’association doit pouvoir modifier le stock afin de signaler la vente ou l’utilisation d’un consommable ou d’un goodies voir même encore le prêt ou l’utilisation de matériel. \newline
  Afin de s’assurer que la gestion du stock est effectuée correctement, un administrateur (membre du Conseil Administratif de l'association) devra comparer chaque semaines le stock réel et celui donné par l’application. \newline
  Ainsi, on met en place un rôle d’administrateur permettant de voir toutes les transactions effectuées mais aussi permettant de modifier le stock et de vérifier son état à différentes étapes (dernière vérification effectuée, ...) \newline
  Ainsi, nous mettons en place un système de connexion et d’identifiants afin de différencier les administrateurs des membres actifs mais aussi pour différencier les membres actifs entre eux afin de savoir, en cas d'erreur, qui a modifié les stocks de quel produit. \newline
  La connexion mise en place, les membres actifs doivent pouvoir visualiser les transactions qu'ils ont éffectués afin détécter en amont des fautes de frappes ou toute erreur évidente.

\section{Cas d'utilisation }
  \subsection{Inscription}
    \begin{description}
      \item[Description textuelle :] Un administrateur doit pouvoir ajouter un nouveau serveur, pour cela il doit lui assigner un login (NomPrenom) ce qui créera un lien unique auquel le serveur pourra rentrer son mot de passe
      \item[Contraintes ]:  
        \begin{itemize}
          \item Chaque serveur doit avoir un login différent
        \end{itemize}
    \end{description}
  \subsection{Connexion}
    \begin{description}
      \item[Description textuelle :] Chaque personne doit d'abord être identifiées avant de pouvoir accéder à la gestion des stocks
      \item[Contraintes ]:  
        \begin{itemize}
          \item Les personnes doivent pouvoir rester connecté entre chaque utilisation
        \end{itemize}
    \end{description}
  \subsection{Création d'un produit}
    \begin{description}
      \item[Description textuelle :] Si un nouveau produit est proposé à la vente où si du nouveau matériel est acheté, il faut pouvoir créer un nouveau produit dont le stock pourra être modifié
      \item[Contraintes ]: 
        \begin{itemize}
          \item On ne peut pas créer un produit qui existe déjà
          \item Pour créer un produit, il faut qu'il ai au moins 1 nom et 1 quantité
        \end{itemize} 
    \end{description}
  \subsection{Ajout/Retrait d'un produit dans le stock}
    \begin{description}
      \item[Description textuelle :] On doit pouvoir modifier la quantité d'un produit dans le stock
      \item[Contraintes ]:
        \begin{itemize}
          \item Le produit doit avoir été créé dans l'application
          \item S'il s'agit d'un retrait, on ne peut pas avoir une quantité négative dans le stock
          \item La modification du stock est mise dans le panier du serveur jusqu'à suppression ou validation
        \end{itemize}
    \end{description}
  \subsection{Supprimer un produit}
    \begin{description}
      \item[Description textuelle :] Un produit doit pouvoir être supprimé si il n'est plus proposé à la vente
      \item[Contraintes ]:  
        \begin{itemize}
          \item Seul l'administrateur peut supprimer un produit
          \item Le produit doit avoir été créé dans l'application
          \item La quantité du produit dans le stock doit être de 0
        \end{itemize}
    \end{description}
  \subsection{Prêt/Utilisation de matériel}
    \begin{description}
      \item[Description textuelle :] Un matériel doit pouvoir être utilisé ou emprunté pendant une période donnée
      \item[Contraintes ]:  
        \begin{itemize}
          \item Le matériel doit avoir été créé dans l'application
          \item Le matériel doit avoir un quantité non nulle en stock
          \item Il faut à la fin de chaque emprunt/utilisation mettre un commentaire concernant les choses notables liées à l'utilisation du matériel
        \end{itemize}
    \end{description}
  \subsection{Validation d'un panier}
    \begin{description}
      \item[Description textuelle :] Chaque serveur doit pouvoir valider son panier avec toutes les transactions qu'il souhaite éffectuer
      \item[Contraintes ]:  
        \begin{itemize}
          \item 
        \end{itemize}
    \end{description}
  \subsection{Vider un panier}
    \begin{description}
      \item[Description textuelle :] Chaque serveur doit pouvoir supprimer toutes leurs transactions de leur panier
      \item[Contraintes ]:  
        \begin{itemize}
          \item 
        \end{itemize}
    \end{description}
  \subsection{Valider une transaction}
    \begin{description}
      \item[Description textuelle :] Un administrateur doit pouvoir, si le stock de l'application ne correspond pas au stock réel, vérifier touts les transactions éffectués et par qui depuis sa dernière validation
      \item[Contraintes ]:  
        \begin{itemize}
          \item 
        \end{itemize}
    \end{description}


\end{document}